\documentclass{article}
\usepackage[margin=0.5in]{geometry}
\usepackage[utf8]{inputenc}

\usepackage{amsmath}
\usepackage{amsthm}
\usepackage{amssymb}
\usepackage{enumerate}
\usepackage{chngcntr}
\usepackage{mathtools}
\usepackage{enumitem}
\usepackage{listings}
\usepackage{hyperref}
\usepackage[dvipsnames]{xcolor}
\usepackage[bb=boondox]{mathalfa}
\newcommand{\Z}{\mathbb{Z}}
\newcommand{\C}{\mathbb{C}}
\newcommand{\HH}{\mathbb{H}}
\newcommand{\Q}{\mathbb{Q}}
\newcommand{\R}{\mathbb{R}}
\newcommand{\N}{\mathbb{N}}
\newcommand{\F}{\mathbb{F}}
\newcommand{\qbinom}{\genfrac{[}{]}{0pt}{}}
\DeclarePairedDelimiter\ceil{\lceil}{\rceil}
\DeclarePairedDelimiter\floor{\lfloor}{\rfloor}

\usepackage[dvipsnames]{xcolor}

\newtheorem{theorem}{Theorem}
\newtheorem{corollary}{Corollary}[theorem] 
\newtheorem{lemma}[theorem]{Lemma} 
\newtheorem{proposition}{Proposition}
\newcommand{\greenparagraph}[1]{\textcolor{ForestGreen}{\textbf{#1}}}

\theoremstyle{definition}
\newtheorem{definition}{Definition}[section]
\theoremstyle{remark}
\newtheorem*{remark}{Remark}
\theoremstyle{remark}
\newtheorem*{note}{Note}
\theoremstyle{definition}
\newtheorem{example}{Example}[definition]
\newcounter{exercise}[subsection]
\newenvironment{exercise}{\refstepcounter{exercise}\textbf{Exercise~\theexercise}}{}
\counterwithin*{equation}{section}
\counterwithin*{equation}{subsection}
\lstset{
  basicstyle=\ttfamily,
  mathescape
}
\author{Rindra Razafy, "Hagamena"}
\title{Borcherd Algebraic Geometry 1}
\setcounter{section}{-1}
\begin{document}

\maketitle
\tableofcontents

\newpage

\section{Prologue}
Some quick notes summarising the "Algebraic geometry 1" video lectures from R.E. Borcherds, found \href{https://youtube.com/playlist?list=PL8yHsr3EFj53j51FG6wCbQKjBgpjKa5PX}{here} 

\newpage

\section{Introduction}
\subsection{Examples}
\subsubsection{Pythagorean triangles}
\textbf{Problem:} How do we classify all Pythagorean triangles.

We will look at two ways of solving this:\begin{enumerate}
    \item \textbf{Algebraic way:} 
    We want to solve\begin{equation}
        x^2+y^2 = z^2 \text{ with }x,y,z\text{ coprime integers}
    \end{equation}
    
    If we look at the equation mod $4$ we notice that $x^2,y^2,z^2 \equiv 0,1 \mod 4$, since the squares mod $4$ all take these forms.
    
    So $z$ is odd and WLOG we assume that $x$ is even and $y$ is odd. 
    We rearange the equation:\begin{equation}
        y^2 = z^2-x^2 = (z-x)(z+x)
    \end{equation}
    Assume that $z-x = dm_1$ and $z+x = dm_2$, therefore we have that $2z = d(m_1+m_2)$ and $2x = d(m_2-m_1)$, then since $d\mid 2z$ and $d\mid 2x$, and $\gcd(x,z) = 1$ we have two cases, either $d$ divides both $x$ and $z$, which would imply that $d=1$.
    
    Or $d$ divides $2$ which means that $d = 1$, or $d=2$. But note that since $x,z$ are of opposite parity $z+x$ is odd so $d\neq 2$.
    
    So in all cases, $d = 1$. So $(z-x)$ and $(z+x)$ are coprime.
    
    But since their product is a square this implies that $z-x$ and $z+x$ are squares. so:\begin{equation}
        z-x = r^2, \text{ and }z+x = s^2, \text{ where }s,r\text{ are odd and coprime}
    \end{equation}
    
    So we conclude that $z = \frac{r^2+s^2}{2}$, $x = \frac{s^2-r^2}{2}, y=rs $ for any $r,s$ odd and coprime.
    
    \item \textbf{Geometric solution}
    Let $X = \frac{x}{z}$, $Y = \frac{y}{z}$ and we want to solve\begin{equation}
        X^2+Y^2=1, \ X,Y\text{rational}
    \end{equation}
    
    So we are looking for rational points on the unit circle.
    
    Note if we draw the line from $(-1,0)$ to $(X,Y)$ on the unit circle with $X,T\in \Q$.It will intersect the y-axis at the point $(0,t)$ where $t=\frac{Y}{X+1}\in \Q$.
    
    Conversely, if we are given $t$ we can find $(X,Y)$, since we know that\begin{equation*}
        Y=t(X+1) \text{ and }t^2(X+1)^2 + X^2 = 1 \Rightarrow (X+1)((t^2+1)X+t^2-1) = 0
    \end{equation*}
    And finding roots we see that $X = \frac{1-t^2}{1+t^2}$ and $Y=\frac{2t}{1+t^2}$, for $t\in \Q$.
    
    \
    
    So there is a correspondence between points on the circle except for the point at $(-1,0)$ and points on the $y-$axis. This is what is called a Birational Equivalence.
    
    \definition\textbf{Birational Equivalence}
    An equivalence excepts on subsets of co-dimension at least $1$.
\end{enumerate}
Treating this problem as a geometrical problem gives us additional insights. Indeed, for example the circle forms a group of rotations with operation:\begin{equation}
    (x_1,y_1)\times (x_2,y_2) = (x_1x_2-y_1y_2,x_1y_2+x_2y_1)
\end{equation}

This is the cosine and sign of the sum of two angles, indeed if $(x_1,y_1) = (\cos\theta_1,\sin\theta_1)\text{ and }(x_2,y_2) = (\cos\theta_2,\sin\theta_2)$ then:\begin{equation}
 (\cos\theta_1,\sin\theta_1)\times (\cos\theta_2,\sin\theta_2) = (\cos\theta_1\cos\theta_2-\sin\theta_1\sin\theta_2, \dots) = (\cos(\theta_1+\theta_2),\sin(\theta_1+\theta_2))
\end{equation}
This is the simplest example of what is called an Algebraic group.

\definition\textbf{Algebraic Groups} We can think of this as functor from (commutative) Rings to Groups.
\begin{equation}
   G\colon R\rightarrow (\{(x,y)\in R^2\mid x^2+y^2=1\},\times)
\end{equation}
Where the operation is defined as above, and the identity is $(1,0)$ and $(x,y)^{-1} = (x,-y)$.

\example $G(\C) = \{(x,y)\in\C \mid x^2+y^2 = 1\}$

But note that $1 = x^2+y^2 = {\underbrace{(x+iy)}_z}{\underbrace{(x-iy)}_{\overline{z}}}$. So we see that \begin{equation}
    G(\C) = \{(x,y)\in\C \mid x^2+y^2 = 1\}\simeq \{z\in \C \mid z\text{ is invertible}\} = \C^\ast
\end{equation}

\textbf{Summary}
There are many ways to view a circle:\begin{enumerate}
    \item Subset of $\R^2$
    \item Polynomial $x^2+y^2-1 \rightarrow$   Algebraic set
    \item Ideal $(x^2+y^2-1)$ in ring $\R[x,y]$.
    \item Ring $\R[x,y]/(x^2+y^2-1) = $ coordinate ring of $S^1$. Can be seen as the set of polynomials on the circle.
    \item (Smooth) manifold
    \item Group (Algebraic Group)
    \item Functor from Rings to Groups or Sets (Grothendieck)
\end{enumerate}
\end{document}
